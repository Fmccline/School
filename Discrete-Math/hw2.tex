\documentclass{article}
\usepackage{setspace}  % To use linespacing
\usepackage{indentfirst} % Indents first line after sections
\usepackage{amssymb} % for \mathbb
% Set up header
\title{Math 307: HW2}
\author{Frank Cline}
\date{\today}
% Make new commands
\newcommand{\N}{\mathbb{N}}
\newcommand{\R}{\mathbb{R}}
\newcommand{\Z}{\mathbb{Z}}
\newcommand{\abs}[1]{\left|#1\right|}



% Start main document
\begin{document}
\onehalfspacing
\hfill Frank Cline

\hfill Math 307

\hfill HW 2

% SECTION 1.6
\section*{1.6} % * after section makes it not numbered

\begin{enumerate}
\setcounter{enumi}{0}
\item OEIS Sequence

PUT ANSWER HERE
\end{enumerate}

% SECTION 1.7
\section*{1.7 :} %  4, 5, 6, 13

\begin{enumerate}
\setcounter{enumi}{3}
\item Consider the following functions from $\N$  into $\N$:\\
$1_\N (n)=n$, $f(n) = 3n$, $g(n) = n+(-1)^n$, $h(n)=min\{n,100\}$, \\$k(n)=max\{0,n-5\}$.
	\begin{enumerate}
	\item Which of these functions are one-to-one?\\
	$f(n)$ is one-to-one because there exists a $3n$ that maps to every $n$.\\
	$g(n)$ is one-to-one because even numbers map to the next odd number, and odd numbers map to the previous even number.\\

	\item Which of these functions map $\N$ onto $\N$\\
	
	\end{enumerate}
\item Here are two "shift functions" mapping $\N$ into $\N$ :\\
$f(n)=n+1$ and $g(n)=max\{0,n-1\}$ for $n \in \N$.
	\begin{enumerate}
	\item Calculate $f(n)$ for $n=0,1,2,3,4,73$.\\

	\item Calculate $g(n)$ for $n=0,1,2,3,4,73$.\\

	\item Show that $f$ is one-to-one but does not map $\N$ onto $\N$.\\

	\item Show that $g$ maps $\N$ onto $\N$ but is not one-to-one.\\

	\item Show that $g(f(n))=n$ for all $n$, but that $f(g(n))=n$ does not hold for all $n$.\\
	
	\end{enumerate}
\item Let $\sum =\{a,b,c\}$ and let $\sum^\ast$ be the set of all words $w$ using the letters from $sum$. Define $L(w)=$length$(w)$ for all $w \in \sum^\ast$.
	\begin{enumerate}
	\item Calculate $L(w)$ for the words $w_1=cab$, $w_2=ababac$, and $w_3=\lambda$.\\

	\item Is $L$ a one-to-one function? Explain.\\

	\item The function $L$ maps $\sum^\ast$ into $\N$. Does $L$ map $\sum^\ast$ onto $\N$? Explain.\\

	\item Find all words $w$ such that $L(w)=2$.\\
	
	\end{enumerate}
\end{enumerate}
\begin{enumerate}
\setcounter{enumi}{12}
\item Let $f : S\to T$ and $g: T\to U$ be one-to-one functions. Show that the function $g\circ f: S\to U$ is one-to-one.
	\begin{itemize}
	\item hello
	\end{itemize}
\end{enumerate}


\end{document}