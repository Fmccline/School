\documentclass{article}
\usepackage{setspace}  % To use linespacing
\usepackage{indentfirst} % Indents first line after sections
\usepackage{amssymb} % for \mathbb
\usepackage{enumerate} %for changing labels of enumerate

% Make new commands
\newcommand{\N}{\mathbb{N}}
\newcommand{\R}{\mathbb{R}}
\newcommand{\Z}{\mathbb{Z}}
\newcommand{\abs}[1]{\left|#1\right|}
\newcommand{\fivespace}{\space\space\space\space\space}

% Start main document
\begin{document}
\onehalfspacing
\hfill Frank Cline

\hfill Math 307

\hfill HW 2

% SECTION 1.6
\section*{1.6} % * after section makes it not numbered

\begin{itemize}
\item OEIS Sequence\\
Sequence of triples: The 10 solutions of $1/p+1/q+1/r=1/2$ with $0<p\leq q\leq r$, lexicographically stored.\\
This sequence is interesting because its first three terms are $3,7,42$ which are three of my favorite numbers.
The sequence also answers an interesting question which is: $1/p+1/q+1/r=1/2$ with $0<p\leq q\leq r$\\
\end{itemize}

% SECTION 1.7
\section*{1.7} %  4, 5, 6, 13

\begin{enumerate}
\setcounter{enumi}{3}
\item Consider the following functions from $\N$  into $\N$:\\
$1_\N (n)=n$, $f(n) = 3n$, $g(n) = n+(-1)^n$, $h(n)=min\{n,100\}$, \\$k(n)=max\{0,n-5\}$.
	\begin{enumerate}
	\item Which of these functions are one-to-one?\\
	$f(n)$ is one-to-one because there exists a $3n$ that maps to every $n$.\\
	$g(n)$ is one-to-one because even numbers map to the next odd number, 
	and odd numbers map to the previous even number.
	\item Which of these functions map $\N$ onto $\N$\\
	$g(n)$ is onto because every even number is mapped to the next odd, 
	and every odd number is mapped to the previous even. Thus all numbers of $\N$ are mapped.\\
	$k(n)$ is onto because all $n$ is either mapped to 0 or $n-5$ which includes all of $\N$.
	\end{enumerate}
\item Here are two "shift functions" mapping $\N$ into $\N$ :\\
$f(n)=n+1$ and $g(n)=max\{0,n-1\}$ for $n \in \N$.
	\begin{enumerate}
	\item Calculate $f(n)$ for $n=0,1,2,3,4,73$.\\
	$f(0)=1$, $f(1)=2$, $f(2)=3$, $f(3)=4$, $f(4)=5$, $f(73)=74$
	\item Calculate $g(n)$ for $n=0,1,2,3,4,73$.\\
	$g(0)=0$, $g(1)=0$, $g(2)=1$, $g(3)=2$, $g(4)=3$, $g(73)=72$
	\item Show that $f$ is one-to-one but does not map $\N$ onto $\N$.\\
	For all $n$, $f(n)$ maps to $n+1$ which is different for all $n$, thus $f(n)$ is one-to-one. 
	When $n=0$, $f(n)=1$ which is the smallest number mapped to by $f(n)$. Thus $0$ isn't included, 
	so all of $\N$ isn't mapped to by $f(n)$. Therefore, $f(n)$ is not onto.
	\item Show that $g$ maps $\N$ onto $\N$ but is not one-to-one.\\
	For all $n$, $g(n)$ maps to $0$ or $n-1$ thus all of $\N$ is mapped to. When $n=0$ and $n=1$, 
	$g(n)=0$ for both, thus $g(n)$ isn't one-to-one.
	\item Show that $g\circ f(n)=n$ for all $n$, but that $f\circ g(n)=n$ does not hold for all $n$.\\
	$g\circ f(n) = max\{0,(n+1)-1\} = max\{0,n\} = n$ for all $n$.\\
	$f\circ g(n) = max\{0,n-1\} + 1$, and when $n=0$, $f\circ g(0)=0+1=1$ and $0\not=1$ thus 
	$f\circ g(n)\not=n$ for all $n$.
	\end{enumerate}
\item Let $\sum =\{a,b,c\}$ and let $\sum^\ast$ be the set of all words $w$ using the letters from $sum$. Define $L(w)=$length$(w)$ for all $w \in \sum^\ast$.
	\begin{enumerate}
	\item Calculate $L(w)$ for the words $w_1=cab$, $w_2=ababac$, and $w_3=\lambda$.\\
	$L(w_1)=3$, $L(w_2)=6$, $L(w_3)=0$.
	\item Is $L$ a one-to-one function? Explain.\\
	No because when $w=bb$ and when $w=aa$, $L(w)$ for both is $2$. Thus $L$ is not one-to-one.
	\item The function $L$ maps $\sum^\ast$ into $\N$. Does $L$ map $\sum^\ast$ onto $\N$? Explain.\\
	$L$ maps $\sum^\ast$ onto $\N$ because $\sum^\ast$ contains the set of all words using \{a,b,c\} 
	which means $L(w)$ can be any positive integer or zero which is all of $\N$. Thus $L$ maps 
	$\sum^\ast$ onto $\N$.
	\item Find all words $w$ such that $L(w)=2$.\\
	$w=aa$, $w=ab$, $w=ac$, $w=bb$, $w=ba$,\\$w=bc$, $w=cc$, $w=ca$, $w=cb$.
	\end{enumerate}
\setcounter{enumi}{12}
\item Let $f : S\to T$ and $g: T\to U$ be one-to-one functions. Show that the function $g\circ f: S\to U$ is one-to-one.
	\begin{itemize}
	\item If $s_1\not=s_2$ then $f(s_1)\not=f(s_2)$ because $f$ is one-to-one. Then $g(f(s_1))\not=
	g(f(s_2))$ because $f(s_1)\not=f(s_2)$ and $g$ is one-to-one. Thus $g\circ f$ is one-to-one.
	\end{itemize}
\end{enumerate}

\section*{2.1} % 2,4,6,7,12,14,16

\begin{enumerate}
\setcounter{enumi}{1}
\item Let $p, q,$ and $r$ be the following prepositions:\\
	$p=$ it is raining\\
	$q=$ the sun is shinging\\
	$r=$ there are clouds in the sky\\
	Translate the following into English sentences.
	\begin{enumerate}
	\item $(p\wedge q)\to r$\\
	If it is raining and it is sunny, then there are clouds in the sky.
	\item $(p\to r)\to q$\\
	If it is raining, then there are clouds in the sky, which implies the sun is shining.
	\item $\neg p\leftrightarrow (q\vee r)$\\
	It is not raining if and only if the sun is shining or only if there are
	clouds in the sky.
	\item $\neg (p\leftrightarrow (q\vee r))$\\
	It is not raining if and only if the sun is not shining or only if there are not clouds in the sky.
	\item $\neg (p\vee q)\wedge r$\\
	It is not raining, or the sun is not shining, and there are clouds in the sky.
	\end{enumerate}
\setcounter{enumi}{3}
\item Which of the following are propositions? Give the truth values of the propositions.
	\begin{enumerate}
	\item $x^2=x$ $\forall x\in\R$.\\
	False
	\item $x^2=x$ for some $x\in\R$.\\
	True
	\item $x^2=x$.\\
	False
	\item $x^2=x$ for exactly one $x\in\R$.\\
	False
	\item $xy=xz$ implies $y=z$.\\
	False
	\item $xy=xz$ implies $y=z$ $\forall x,y,z\in\R$.\\
	False
	\item $w_1w_2=w_1w_3$ implies $w_2=w_3$ for all words $w_1,w_2,w_3\in\sum^\ast$.\\
	True
	\end{enumerate}
\setcounter{enumi}{5}
\item Give the converses of the following propositions.
	\begin{enumerate}
	\item $q\to r$.\\
	$r\to q$
	\item If I am smart, then I am rich.\\
	If I am rich, then I am smart.
	\item If $x^2=x$, then $x=0$ or $x=1$.\\
	If $x=0$ or $x=1$, then $x^2=x$.
	\item If $2+2=4$, then $2+4=8$.\\
	If $2+4=8$, then $2+2=4$.
	\end{enumerate}
\item Give the contrapositives of the propositions in exercise 6.
	\begin{enumerate}
	\item $q\to r$.\\
	$\neg r\to \neg q$
	\item If I am smart, then I am rich.\\
	If I am not rich, then I am not smart.
	\item If $x^2=x$, then $x=0$ or $x=1$.\\
	If $x\not=0$ or $x\not=1$, then $x^2\not=x$.
	\item If $2+2=4$, then $2+4=8$.\\
	If $2+4\not=8$, then $2+2\not=4$.
	\end{enumerate}
\setcounter{enumi}{11}
\item Find counterexamples to the following assertions.
	\begin{enumerate}
	\item $2^n+1$ is prime for every $n\geq2$.\\
	When $n=3$, $2^3+1=9$ and $3\mid9$ so $9$ is not prime.
	\item $2^n+3^n$ is prime $\forall n\in\N$.\\
	When $n=3$, $2^3+3^3=35$ and $5\mid35$ so $35$ is not prime. 
	\item $2^n+n$ is prime for every positive integer $n$.\\
	When $n=4$, $2^4+4=20$ and $5\mid20$ so $20$ is not prime.
	\end{enumerate}
\setcounter{enumi}{13}
\item Let $S$ be a nonempty set. Determine which of the following assertions are true.
For the true ones, give a reason. For the false ones, provide counterexamples.\\
	\begin{enumerate}
	\item $A\cup B=B\cup A$\fivespace$\forall A,B\in P(S).$\\
	True. $A\cup B$ is the set of all elements in $A$ and of all elements in $B$. $B\cup A$ is the set of
	all the elements in $B$ and all of the elements in $A$ which is equal to $A\cup B$.
	\item $(A\setminus B)\cup B=A$\fivespace$\forall A,B\in P(S).$\\
	False. If $S = \{1,2,3,4\}$ $\exists A=\{1,2,3\}$ and $\exists B=\{2,3,4\}$ in $P(S)$. 
	$(A\setminus B)\cup B=\{1,2,3,4\}\not=A$. Thus the statement is false.
	\item $(A\cup B)\setminus A=B$\fivespace$\forall A,B\in P(S).$\\
	True. $(A\cup B)$ is the set of all elements in $A$ and all elements in $B$. $(A\cup B)\setminus A$ 
	is all of the elements in $A$ and all of the elements in $B$ minus all of the elements in $A$ which leaves 
	all of the elements in $B$. Thus the statement is True.
	\item $(A\cap B)\cap C=A\cap(B\cap C)$\fivespace$\forall A,B\in P(S).$\\
	True. An element in set of all elements in $A$ and $B$ that are also in $C$ will also be in the 
	set of all elements in B and C that are also in A.
	\end{enumerate}
\setcounter{enumi}{15}
\item Consider the following propositions: $r="ODD(N)=T"$, $m=$"the output goes to the monitor", and 
$p=$"the output goes to the printer". Translate each of the following propositions into symbols.
	\begin{enumerate}
	\item The output goes to the monitor if $ODD(N)=T$.\\
	$r\to m$
	\item The output goes to the printer whenever $ODD(N)=T$ is not true.\\
	$\neg r\to p$
	\item $ODD(N)=T$ only if the output goes to the monitor.\\
	$r\leftrightarrow m$
	\item The output goes to the monitor if the output goes to the printer.\\
	$p\to m$
	\item $ODD(N)=T$ or the output goes to the monitor if the output goes to the printer.\\
	$p\to (r\vee m)$
	\end{enumerate}	
\end{enumerate}

\section*{2.2} % 2,3,12,23,24
\begin{enumerate}
\setcounter{enumi}{1}
\item Consider the propostion "if $x>0$, then $x^2>0$." Here $x\in\R$
	\begin{enumerate}
	\item Give the converse and contrapositive of the proposition.\\
	Converse: if $x^2>0$ then $x>0$.\\
	Contrapositive: if $x^2\leq0$ then $x\leq0$.
	\item Which of the following are true propositions: the original, its converse, or its contrapositive?\\
	The original is true and the contrapositive is true.
	\end{enumerate}
\item Consider the following propositions:\\
$p\to q$, $\neg p\to\neg q$, $q\to p$, $\neg q\to\neg p,$\\
$q\wedge\neg p$, $\neg p\vee q$, $\neg q\vee p$, $p\wedge\neg q$.
	\begin{enumerate}
	\item Which proposition is the converse of $p\to q$?\\
	$q\to p$
	\item Which proposition is the contrapositive of $p\to q$?\\
	$\neg q\to\neg p$
	\item Which propositions are logically equivelent to $p\to q$?\\
	$\neg q\to\neg p$, $\neg p\vee q$
	\end{enumerate}
\setcounter{enumi}{11}
\item In which of the following statements is the "or" an "inclusive or"?
	\begin{enumerate}
	\item Choice of soup or salad.\\
	This "or" is not inclusive.
	\item To enter the university, a student must have taken a year of
	chemistry or physics in high school.\\
	This "or" is inclusive.
	\item Publish or perish.\\
	This "or" is not inclusive.
	\item Experience with C++ or Java is desirable.\\
	This "or" is inclusive.
	\item The task will be completed on Thursday or Friday.\\
	This "or" is not inclusive.
	\item Discounts are available to persons under 20 or over 60.\\
	This "or" is not inclusive.
	\item No fishing or hunting allowed.\\
	This "or" is inclusive.
	\item The school will not be open in July or August.\\
	This "or" is inclusive.
	\end{enumerate}
\setcounter{enumi}{22}
\item A logician told her son "If you don't finish your dinner, you will not get to 
stay up and watch TV." He finished his dinner and then was strent straight to bed. Discuss.
	\begin{itemize}
	\item Let $f=$"finished dinner" and let $s=$"sent to bed". $f$ is true and $s$ is true 
	so the logician's statement of $\neg f\to\neg s$ is logically correct.
	\end{itemize}
\item Consider the statement "Concrete does not grow if you do not water it."
	\begin{enumerate}
	\item Give the contrapositive.\\
	If you water concrete, it will grow.
	\item Give the converse.\\
	If you you do not water concrete, it will not grow.
	\item Give the converse of the contrapositive.\\
	Concrete will grow if you water it
	\item Which among the original statement and the ones in parts a, b, and c are true?\\
	All the statements are true.
	\end{enumerate}
\end{enumerate}

\section*{Other Exercises} For the folowing statements:
\begin{enumerate}[i.]
\item Translate into symbols
\item Write the negation of the statement in words
\item Write the contrapositive of the original statement in words
\end{enumerate}
\begin{enumerate}
\item For all the functions $f:S\to T$, if $f:S\to T$ is onto, then for all $s\in S$ 
there exists a $t\in T$ such that $f(s)=t$.
	\begin{enumerate}[i.]
	\item Let $f=(f:S\to T) $ and an onto function.\\$\forall f\to(\forall s\in S$, $\exists t\in T:f(s)=t)$.
	\item For some function $f:S\to T$ that is onto, then there exists a $s\in S$ for all $t\in T$ that $f(s)\not=t$.
	\item There exists a $s\in S$ for all $t\in T$ such that $f(s)\not=t$ for some function $f:S\to T$ that is not onto.
	\end{enumerate}
\item For all graphs $G$, if $G$ is finite and connected, then $G$ has a spanning tree.
	\begin{enumerate}[i.]
	\item Let $gf=$ a graph that is finite, $gc=$ a graph that is connected, $s=$ a graph with a spanning tree.\\
	$\forall (G:gf\wedge gc)$, $G\to s$
	\item For some graph $G$, if $G$ is not finite or $G$ is not connected, then $G$ does not have a spanning tree.
	\item For some graph $G$, if $G$ does not have a spanning tree, then $G$ is not finite and connected.
	\end{enumerate}
\item If G is a finite connected graph and every vertex has even degree, then G is Eulerian.
	\begin{enumerate}[i.]
	\item Let $gf=$ a graph that is finite, $gc=$ a graph that is connected, $ged=$ a graph with vertices with even degrees, 
	and $e=$ a Eulerian graph.
	$(G\wedge gf\wedge gc\wedge ged)\to e$
	\item If $G$ is not a finite connected graph or every vertex does not have even degrees, then $G$ is not Eulerian.
	\item If $G$ is not Eulerian, then $G$ is not a finite connected graph or every vertex does not have even degrees.
	\end{enumerate}
\item For all graphs $G$ with no loops or parallel edges, if $\abs{V(G)}=n\geq3$ and $deg(v)\geq n/2$ for each vertex of $G$, 
then $G$ is Hamiltonian.
	\begin{enumerate}[i.]
	\item Let $gl=$ a graph with loops, $gpe=$ a graph with parallel edges, $v=$ vertex of a graph, $h=$ a Hamiltonian graph.\\
	$\forall G\wedge\neg(gl\wedge gpe)$, $(\abs{V(G)}=n\geq3)\wedge(deg(v)\geq n/2)$ $\forall v\in G\to h$.
	\item There exists a graph $G$ with loops and parallel edges, if $\abs{V(G)}\not=n\geq3$ and $deg(v)<n/2$ for some vertex of 
	$G$, then $G$ is not Hamiltonian.
	\item If $G$ is not Hamiltonian, then $G$ has loops and parallel edges with $\abs{V(G)}\not=n\geq3$ and $deg(v)<n/2$ 
	for some vertex of $G$.
	\end{enumerate}
\end{enumerate}






\end{document}