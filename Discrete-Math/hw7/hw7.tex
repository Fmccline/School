\documentclass[12pt]{article}
\usepackage{setspace}  % To use linespacing
\usepackage{indentfirst} % Indents first line after sections
\usepackage{amssymb} % For \mathbb
\usepackage{enumerate} % For changing labels of enumerate
\usepackage[margin=1in]{geometry} % For editing margins
\usepackage{tikz} % Tikz drawing for graphs
\usetikzlibrary{arrows.meta} % Allows customizing arrows
\usetikzlibrary{backgrounds} % For framing a tikzpicture
\usepackage{amsmath}

% Make new commands
\newcommand{\N}{\mathbb{N}}
\newcommand{\R}{\mathbb{R}}
\newcommand{\Z}{\mathbb{Z}}
\newcommand{\abs}[1]{\left|#1\right|}
\newcommand{\paren}[1]{\left(#1\right)}
\newcommand{\fivespace}{\space\space\space\space\space}

\newcommand{\be}{\begin{enumerate}}
\newcommand{\ee}{\end{enumerate}}
\newcommand{\seti}[1]{\setcounter{enumi}{#1}}
\newcommand{\setii}[1]{\setcounter{enumii}{#1}}

% Start main document
\begin{document}
\onehalfspacing
\hfill Frank Cline

\hfill Math 307

\hfill HW 7

% Sections
% 6.3 # 2, 8, 11a
% 6.4 # 6, 8, 10, 12 and additional problems
% 6.5 # 2, 4, 8
% 4.6 # 8

% SECTION 6.3 # 2, 8, 11a
\section*{6.3}
\be

% 8
\seti{7}
\item Consider a tree with $n$ vertices. It has exactly $n-1$ edges [Lemma 2], so the sum of its of the degrees of its vertices is $2n-2$.
	\be
	\item A tree has two vertices of degree 5, three of degree 3, two of degree 2, and the rest of degree 1. 	How many vertices are in the graph?\\
	
	$2|E(G)| = 2n-2 = 2(5)+3(3)+2(2)+x$
	
	\ee

% 11a
\seti{10}
\item
	\be
	\item Show that a forest with $n$ vertices and $m$ components has $n-m$ edges.\\
	
	
	\ee
\ee

\end{document}