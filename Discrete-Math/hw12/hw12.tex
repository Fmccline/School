\documentclass[12pt]{article}
\usepackage{setspace}  % To use linespacing
\usepackage{indentfirst} % Indents first line after sections
\usepackage{amssymb} % For \mathbb
\usepackage{enumerate} % For changing labels of enumerate
\usepackage[margin=1in]{geometry} % For editing margins
\usepackage{tikz} % Tikz drawing for graphs
\usetikzlibrary{arrows.meta} % Allows customizing arrows
\usetikzlibrary{backgrounds} % For framing a tikzpicture
\usetikzlibrary{calc, through}
\usetikzlibrary{decorations.markings}
\usetikzlibrary{arrows}
\usetikzlibrary{positioning}
\usepackage{amsmath}
\usepackage{ifthen}
\usepackage{intcalc} % \intcalcMod

% Make new commands
\newcommand{\N}{\mathbb{N}}
\newcommand{\R}{\mathbb{R}}
\newcommand{\Z}{\mathbb{Z}}
\newcommand{\abs}[1]{\left|#1\right|}
\newcommand{\paren}[1]{\left(#1\right)}
\newcommand{\fivespace}{\space\space\space\space\space}

\newcommand{\be}{\begin{enumerate}}
\newcommand{\ee}{\end{enumerate}}
\newcommand{\seti}[1]{\setcounter{enumi}{#1}}
\newcommand{\setii}[1]{\setcounter{enumii}{#1}}
\newcommand{\tand}{\text{ and }}
\newcommand{\floor}[2]{\left\lfloor\frac{#1}{#2}\right\rfloor}


% Start main document
\begin{document}
\onehalfspacing
\hfill Frank Cline

\hfill Math 307

\hfill HW 12

% 5.3 #2, 4, 9cd, 11a, 12, 13bc
% 5.5 # 2, 4, 6, 13
% 5 Supplementary Exercises # 11, 39

\section*{5.3}
\be

% 2
\seti{1}
\item Let $S=\{100,101,102,...,999\}$ so that $|S|=900$.
	\be
	% 2 a
	\item How many numbers in $S$ have at least one digit that is a 3 or a 7?\\
	First, find the amount of numbers that don't have a 3 or a 7 in them. That amount is equal to $7\choose1$ for the first 
	digit because it can be numbers 1-9 excluding 3 and 7. The second two digits can be numbers 0-9 exculding 3 and 7, so that is 
	$8\choose1$. Thus, the equation for numbers that don't have a single digit as 3 or 7 is ${7\choose1}{8\choose1}{8\choose1}=
	448$. The amount of numbers that have at least one 3 or 7 is equal to the total amount of numbers minus the amount of numbers 
	that don't have a 3 or a 7 in them which equals $900-448=454$.

	% 2 b
	\item How many numbers in $S$ have at least one digit that is a 3 and at least one digit that is a 7?\\
	To find the amount of numbers that have at least one digit that is 3 and one digit that is 7, first choose 2 of the 3 digits 
	to be 3 and 7 where order matters $P(3,2)$. Then choose which  Then pick the last number's digit which can be between 0-9 
	$10\choose1$. The total is equal to $P(3,2)\times{10\choose1}=(6)(10)=60$.
	\ee
	
% 4
\seti{3}
\item An investor has 7 \$1000 bills to distribute among 3 mutual friends.
	\be
	% 4 a
	\item In how many ways can she invest her money?\\
	
	
	% 4 b
	\item In how many ways can she invest her money if each fund must get at least \$1000.\\
	
	
	\ee

% 9 c d
\seti{8}
\item Use the binomial theorem to expand the following:
	\be
	\setii{2}
	% 9 c
	\item $(3x+1)^4$\\
	\begin{align*}
		(3x+1)^4 &= \sum^4_{r=0}{4\choose r}(3x)^r(1)^{4-r}\\
				&= {4\choose0}(3x)^0(1)^{4-0} + {4\choose1}(3x)^1(1)^{4-1} + {4\choose2}(3x)^2(1)^{4-2}\\ &+
				 {4\choose3}(3x)^3(1)^{4-3} + {4\choose4}(3x)^4(1)^{4-4}\\
				&= 1 + (4)(3x)^1 + (6)(3x)^2 + (6)(3x)^3 + (4)(3x)^4 \\
				&= 1 + 12x + 54x^2 + 162x^3 + 324x^4
	\end{align*}
	% 9 d
	\item $(x+2)^5$\\
	\begin{align*}
		(x+2)^5 &= \sum^5_{r=0}{5\choose r}(x)^r(2)^{5-r}\\
				&= {5\choose0}(x)^0(2)^{5-0} + {5\choose1}(x)^1(2)^{5-1} + {5\choose2}(x)^2(2)^{5-2} \\&+ 
					{5\choose3}(x)^3(2)^{5-3} + {5\choose4}(x)^4(2)^{5-4} + {5\choose5}(x)^5(2)^{5-5}\\
				&= (2)^5 + 5x(2)^4 + 10x^2(2)^3 + 10x^3(2)^2 + 5x^4(2)^1 + x^5 \\
				&= 32 + 80x + 80x^2 + 40x^3 + 10x^4 + x^5
	\end{align*}
	
	\ee
	

% 11 a
\seti{10}
\item Prove that $2^n = \sum^n_{r=0}{n\choose r}$
	\be
	\item by setting $a=b=1$ in the binomial theorem.
	\begin{align*}
		(2)^n &= (1+1)^n \\
		(1+1)^n &= \sum^n_{r=0}{n\choose r}(1)^r(1)^{n-r} \hspace{1cm}\text{(by the binomial theorem)}\\
				&= \sum^n_{r=0}{n\choose r}(1)(1) \hspace{1cm}\text{(because 1 to the power of anything is still 1)}\\
				&= \sum^n_{r=0}{n\choose r}
	\end{align*}
	Thus $2^n = \sum^n_{r=0}{n\choose r}$. 
	\ee

% 12
\item Prove that $\sum^n_{r=0}{{n\choose r}2^r} = 3^n$ for all $n\in\Z^+$.\\
	By setting a = 2 and b = 1 in the binomial theorem we get:
	\begin{align*}
		(3)^n &= (2+1)^n \\
		(2+1)^n &= \sum^n_{r=0}{n\choose r}(2)^r(1)^{n-r} \hspace{1cm}\text{(by the binomial theorem)}\\
				&= \sum^n_{r=0}{n\choose r}(2)^r(1) \hspace{1cm}\text{(because 1 to the power of anything is still 1)}\\
				&= \sum^n_{r=0}{n\choose r}(2)^r
	\end{align*}
	Thus $3^n = \sum^n_{r=0}{{n\choose r}2^r}$ for all $n\in\Z^+$.

% 13 b c
\item 
	\be
	% 13 b
	\setii{1}
	\item Prove $\sum^n_{k=m}{k\choose m} = {{n+1}\choose{m+1}}$ by induction on $n$ for $n\geq m$.	\\
	\begin{itemize}
		\item (Base Case): \vspace{0.1cm} \\
			The base case is when $n=m$. $\sum^n_{k=m}{k\choose m} = {n\choose m} = {m\choose m} = 1. \\
			{{n+1}\choose{m+1}} = {{m+1}\choose{m+1}} = 1$. 1 = 1 so our base case holds.
		
		\item (Ind Hyp):\vspace{0.1cm} \\ 
			$\sum^{n-1}_{k=m}{k\choose m} = {{n-1+1}\choose{m+1}}$ for some $n>m$.

		\item (Ind Step):
		\begin{align}
			\sum^{n}_{k=m}{k\choose m} &= \sum^{n-1}_{k=m}{k\choose m} + {n\choose m}\\
									&= {{n-1+1}\choose{m+1}} + {n\choose m} \\
									&= {{n}\choose{m+1}} + {n\choose m} \hspace{1.5cm}\text{(By Ind Hyp)}\\
									&= {{n+1}\choose{m+1}}
		\end{align}
	\end{itemize}
		Thus by induction, $\sum^n_{k=m}{k\choose m} = {{n+1}\choose{m+1}}$
	\ee
 
\ee


% 5.5 # 2, 4, 6, 13
\section*{5.5}

\be

% 2
\seti{1}
\item 
	\be
	% 2 a
	\item A sack contains 50 marbles of 4 different colors. Explain why there are at least 13 marbles of the same color.\\
		4 different colors $\to$ 4 boxes. 50 marbles $\to$ 50 balls. $(4)(12)=48<50$, so there are more than 12 marbles 
		($\geq13$) of the same color by the pigeonhole principle.
		
	% 2 b
	\item If exactly 8 marbles are red, explain why there are at least 14 marbles of the same color.\\
		If 8 marbles are red, that leaves 42 marbles of 3 different colors. \\3 different colors $\to$ boxes. 42 marbles $\to$ 
		balls. $(3)(13)=39<42$, so there must be more than 13 marbles ($\geq14$) that are the same color by the pigeonhole 
		principle.
	\ee

% 4
\seti{3}
\item 
	\be
	% 4 a
	\item Let $B$ be a 12-element subset of $\{1,2,3,4,5,6\}\times\{1,2,3,4,5,6\}$. Show that $B$ containts two different 
	ordered pairs, the sums of whose entries are equal.\\
	$(2,3)\in B\tand(1,4)\in B$. $2+3=5\tand1+4=5$, so their sums are equal.

	% 4 b
	\item How many times can a pair of dice be tossed without obtaining the same sum twice?\\
	There 11 different sums (2,3,4,5,6,7,8,9,10,11,12) $\to$ boxes. So a pair of dice can be tossed \fbox{11} times before 
	obtaining the same sum twice.
	\ee

% 6
\seti{5}
\item Let $S$ be a 3-element set of integers. Show that $S$ has two different nonempty subsets such that the sums of the numbers in each of the subsets are congruent mod 6.\\
Congruent mod 6 has 6 equivalence classes (0,1,2,3,4,5) $\to$ boxes. A 3-element set of integers has $|P(S)|$ subets. $|P(S)| = 2^{|S|} = 2^3 = 8 \to \text{ balls}$. There are 7 balls (8 minus the empty set) and only 6 boxes, so there must be at least one box with 2 or more balls. This means that there must be at least 2 different nonempty subsets of $S$ such that their sums are congruent mod 6.

% 13
\seti{12}
\item Let $n_1,n_2,\tand n_3$ be distinct positive integers. Show that at least one of $n_1,n_2,n_3, n_1+n_2,n_2+n_3,n_1+n_2+n_3$ is divisible by 3.


\ee


% 5 Supplementary Exercises # 11, 39
\section*{Supplementary Exercises}
\be
% 11
\seti{10}
\item A box contains tickets of 4 colors: red, blue, yellow, and green. Each ticket has a number from \{0,1,...,9\} written on it.
	\be
	% 11 a
	\item What is the largest number of tickets the box can contain without having at least 21 tickets of the same color?\\
	4 colors $\to$ boxes. $21-1=20\to$ balls. $(4)(20)=80$ so the box can contain \fbox{80} tickets without there being a box with at 
	least 21 tickets of the same color by the pigeonhole principle.

	% 11 b
	\item What is the smallest number of tickets the box must contain to be sure that it contains at least two tickets of the 	
	same color with the same number on them?\\
	4 colors of 10 different numbers is $(4)(10)=40\to$ boxes. $2-1=1\to$ balls. $(40)(1)=40$ so there must be at least \fbox{41}
	tickets for there to be at least 2 tickets with the same color and number on them.
	\ee

% 39
\seti{38}
\item Let $D=\{1,2,3,4,5\}\text{ and let } L=\{a,b,c\}$.
	\be
	% 39 a
	\item How many functions are there from the set $D$ to the set $L$?\\
	There are 5 elements in $D$ each of which can map to one of 3 elements in $L$. Thus, the total number of functions is equal 
	to ${3\choose1}{3\choose1}{3\choose1}{3\choose1}{3\choose1}=\fbox{243}$ functions.

	% 39 b
	\item How many of these functions map $D$ onto $L$?\\
	For $D$ to map ont $L$, all of the elements in $L$ must be mapped to. To do so, first choose 3 of the 5 elements in $D$; 
	these elements will map to the 3 elements in $L$. Then the other 2 elements in $D$ can be mapped to any of the elements in 
	$L$. Thus, the equation is ${5\choose3}{3\choose1}{3\choose1}=\fbox{90}$ of the functions are onto.

	\ee

\ee


\end{document}















