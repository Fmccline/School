\documentclass[12pt]{article}
\usepackage{setspace}  % To use linespacing
\usepackage{indentfirst} % Indents first line after sections
\usepackage{amssymb} % For \mathbb
\usepackage{enumerate} % For changing labels of enumerate
\usepackage[margin=1in]{geometry} % For editing margins
\usepackage{tikz} % Tikz drawing for graphs
\usetikzlibrary{arrows.meta} % Allows customizing arrows
\usetikzlibrary{backgrounds} % For framing a tikzpicture
\usetikzlibrary{calc, through}
\usetikzlibrary{decorations.markings}
\usetikzlibrary{arrows}
\usetikzlibrary{positioning}
\usepackage{amsmath}
\usepackage{ifthen}
\usepackage{intcalc} % \intcalcMod

% Make new commands
\newcommand{\N}{\mathbb{N}}
\newcommand{\R}{\mathbb{R}}
\newcommand{\Z}{\mathbb{Z}}
\newcommand{\abs}[1]{\left|#1\right|}
\newcommand{\paren}[1]{\left(#1\right)}
\newcommand{\fivespace}{\space\space\space\space\space}

\newcommand{\be}{\begin{enumerate}}
\newcommand{\ee}{\end{enumerate}}
\newcommand{\seti}[1]{\setcounter{enumi}{#1}}
\newcommand{\setii}[1]{\setcounter{enumii}{#1}}
\newcommand{\tand}{\text{ and }}
\newcommand{\floor}[2]{\left\lfloor\frac{#1}{#2}\right\rfloor}

% Start main document
\begin{document}
\onehalfspacing
\hfill Frank Cline

\hfill Math 307

\hfill HW 11

% 5.1 # 4bcd (don't actually list them all), 6, 8, 9, 10, 13, 14, 16
% 5.2 # 6, 12, 15, 16, 20

\section*{5.1}

\be
% 4
\seti{3}
\item Let $A=\{1,2,3,4,5,6,7,8,9,10\}\tand B=\{2,3,5,7,11,13,17,19\}$.
	\be
	% 4 b
	\setii{1}
	\item How many subsets of A are there?\\
	There are 11 possible combinations: \\
	The empty set, set of length 1, set of length 2, ... , set of length 10\\
	$={10\choose0}\foreach \i in {1,2,3,4,5,6,7,8,9,10}{+{10\choose\i}}$\\
	$=1+10+45+120+210+252+210+120+45+10+1\\=1024$
	% 4 c
	\item How many 4-element subsets of A are there?\\
	The amount of 4-element subsets of A is equal to ${10\choose4} =210$.
	% 4 d
	\item How many 4-element subsets of A consist of 3 even and 1 odd number?\\
	Three of the elements must be even, and since there are 5 even numbers total in A, an even number is given by ${10\choose5}$.
	Since there are 3 even numbers, and numbers can't be repeated, that is ${10\choose5}{10\choose4}{10\choose3}$. The remaining 
	element must be odd, and there are 5 elements in A that are odd. Thus the remaining element must be ${10\choose5}$. 
	That makes the total to be ${10\choose5}{10\choose5}{10\choose4}{10\choose3}=834$.
	\ee

% 6
\seti{5}
\item A certain class consists of 12 men and 16 women. How many committees can be chosen from this class consisting of
	\be
	% 6 a
	\item 7 people?\\
	This is equal to $28\choose7$ because there are 28 people total, you need to choose 7 of them, and order doesn't matter.
	${28\choose7}=1,184,040$
	% 6 b
	\item 3 men and 4 women\\
	This is equal to $12\choose3$$16\choose4$ because you need to choose 3 out of 12 men and 4 out of 16 women, and order doesn't 
	matter. $12\choose3$$16\choose4$$=400400$
	% 6 c
	\item 7 women or 7 men\\
	This is equal to ${12\choose7}+{16\choose7}$ because there are two possibilities: choosing 7 men OR choosing 7 women where 
	order doesn't matter. ${12\choose7}+{16\choose7}=12,232$
	\ee

% 8
\seti{7}
\item How many committees consisting of 4 men and 4 women can be chosen from a group of 8 men and 6 women?\\
	This is equal to ${8\choose4}{6\choose4}$ because order doesn't matter and we're choosing 4 of 8 men, and 4 of 6 women.
	${8\choose4}{6\choose4}=85$.

% 9
\item Let $S=\{a,b,c,d\}\tand T=\{1,2,3,4,5,6,7\}$.
	\be
	% 9 a
	\item How many one-to-one functions are there from T into S?\\
	There are no one-to-one functions from T into S because T has more elements than S does, thus, it's impossible to map each 
	element in T to a unique element in S which is the definition of one-to-one.
	% 9 b
	\item How many one-to-one functions are there from S into T?\\
	The first element can map to one of 7 elements in T. The second element can then map to one of the 7 elements in T excluding 		the element one is mapped to, and so on. Thus the number of different one-to-one functions for S into T is 
	$(7)(7-1)(7-2)(7-3)=(7)(6)(5)(4)=840$.
	% 9 c
	\item How many functions are there from S into T?\\
	The number of functions from S into is equal to $(7)(7)(7)(7)$ because each of the 4 elements in S can map to any one of the 
	7 elements in T. $(7)(7)(7)(7)=2401$
	\ee

% 10
\item Let $P=\{1,2,3,4,5,6,7,8,9\}\tand Q=\{A,B,C,D,E\}$.
	\be
	% 10 a
	\item How many 4-element subsets of P are there?\\
	The number of 4-element subsets of P is equal to ${9\choose4}$ because you're choosing 4 elements from P and order doesn't 
	matter. ${9\choose4}=126$.
	% 10 b
	\item How many permutations of Q are there?\\
	There are 5 elements in Q, so the number of permutations of Q is equal to $5!=120$. 
	% 10 c
	\item How many license plates are there consisting of 3 letters from Q followed by 2 numbers from P? Repetition is allowed.\\
	The cases are:
		\begin{itemize}
		\item No letter rep: $P(5,3)$ \\
			Find the permutations of size 3 from the 5 letters.
		\item 2 letter rep: ${5\choose1}{4\choose1}{2\choose1}$ \\
			Choose 1 of the 5 letters, then choose another of the remaining 4 letters. The final letter must be the same as one
			of the first 2 letters.
		\item 3 letter rep: ${5\choose1}{1\choose1}{1\choose1}$\\	
			Choose 1 of the 5 letters, and the other 2 must be that letter.
		\item No num rep: $P(9,2)$\\
			Find the permutations of size 2 from the 9 numbers.
		\item 2 num rep: $9\choose1$\\
			Choose 1 of the numbers and they will be repeated.
		\end{itemize}
		The total number of possibilities\\
		$=[(\text{No letter rep})+(\text{2 letter rep})+(\text{3 letter rep})]\times[(\text{no num rep})+(\text{2 num rep})]$ \\
		$=[P(5,3)+(5)(4)(2)+(5)]\times[P(9,2)+9]$\\
		$=(60+40+5)\times(72+9)=(105)\times(81)$\\
		$=8505$.
	\ee

% 13
\seti{12}
\item Let $S$ be the set of 7-digit numbers $\{n\in\N:10^6\leq n<10^7\}$.
	\be
	% 13 a
	\item Find $|S|$\\
	$|S|=10^7-10^6=9,000,000$
	% 13 b
	\item How many members of $S$ are odd?\\
	Each member of S has 7 digits, and for it to be odd, the last digit must odd 
	(and there are 5 odd digits). Thus the equation is:\\
	${9\choose1}{10\choose1}{10\choose1}{10\choose1}{10\choose1}{10\choose1}{5\choose1}$
	$=4,500,000$
	% 13 c
	\item How many are even?\\
	Each member of S has 7 digits, and for it to be even, the last digit must even 
	(and there are 5 even digits). Thus the equation is:\\
	${9\choose1}{10\choose1}{10\choose1}{10\choose1}{10\choose1}{10\choose1}{5\choose1}$
	$=4,500,000$
	% 13 d
	\item How many are multiples of 5?\\
	For a member of S to be a multiple of 5, it's last digit must be either 0 or 5. Thus the equation is:\\
	${9\choose1}{10\choose1}{10\choose1}{10\choose1}{10\choose1}{10\choose1}{2\choose1}$
	$=1,800,000$	
	% 13 e
	\item How many have no two digits the same?\\
	For no two digits to be the same, all the digits must be distinct. The first digit is ${9\choose1}$ 
	because it can be numbers 1-9. The second is also ${9\choose1}$ because it can be numbers
	0-9 excluding the number of the first digit. The digits after that choose from the previous set minus one.
	\\
	${9\choose1}{9\choose1}{8\choose1}{7\choose1}{6\choose1}{5\choose1}{4\choose1}$
	$=544,320$
	% 13 f
	\item How many odd members of S have no two digits the same?\\
	Starting with the last digit, it must be an odd number, so its $5\choose1$. The first digit must then be 
	between 1-9 excluding the last digit. After that, each digit chooses from the previous set minus one.\\
	${5\choose1}{8\choose1}{8\choose1}{7\choose1}{6\choose1}{5\choose1}{4\choose1}$
	$=268,800$
	% 13 g
	\item How many even members of S have no two digits the same?\\
	Starting with the last digit, it must be an even number, so its $5\choose1$. The first digit must then be 
	between 1-9 excluding the last digit if the last digit isn't zero. After that, each digit chooses from the 
	previous set minus one. There's also the special case where the last digit is equal to zero. Then the 
	first digit can be number 1-9. The equation for the number of even numbers with no two digits the 
	same is:\\
	${4\choose1}{8\choose1}{8\choose1}{7\choose1}{6\choose1}{5\choose1}{4\choose1}+
	{1\choose1}{9\choose1}{8\choose1}{7\choose1}{6\choose1}{5\choose1}{4\choose1}
	=275,520$
	\ee
	
% 14
\item Let $S$ be the set of integers between 1 and 10,000.
	\be
	% 14 a
	\item How many members of $S$ are multiples of 3 and also multiples of 7?\\
	$\floor{10000}{21}=476$
	% 14 b
	\item How many members of $S$ are multiples of 3 and also multiples of 7 or multiples of 3 or of 7?\\
	$\floor{10000}{3}+\floor{10000}{7}-\floor{10000}{21}=4285$
	% 14 c
	\item How many members of $S$ are not divisible by 3 or 7?\\
	$10000-\left(\floor{10000}{3}+\floor{10000}{7}\right)=5239$
	% 14 d
	\item How many members of $S$ are divisible by 3 or 7 but not by both?\\
	$\floor{10000}{3}+\floor{10000}{7}-(2)\floor{10000}{21}=3809$
	\ee

%16
\seti{15}
\item 
	\be
	% 16 a
	\item In how many ways can the letters $a,b,c,d,e,f$ be arranged so that the letters $a\tand b$ 
	are right next to each other?\\
	Every spot can be different except when one spot is either $a\text{ or }b$, then the next spot must be 
	the other. Thus, the equation is 
	${2\choose1}{1\choose1}{4\choose1}{3\choose1}{2\choose1}{1\choose1}=48$
	% 16 b
	\item In how many ways can the letters $a,b,c,d,e,f$ be arranged so that the letters $a\tand b$ 
	are not right next to each other?\\
	Every spot can be different except when one spot is either $a\text{ or }b$, then the next spot must be 
	something other than $a\text{ or }b$. Thus, the equation is\\
	${2\choose1}{4\choose1}{4\choose1}{3\choose1}{2\choose1}{1\choose1}=192$
	% 16 c
	\item In how many ways can the letters $a,b,c,d,e,f$ be arranged so that the letters $a\tand b$ 
	are right next to each other but $a\tand c$ aren't?\\
	Every spot can be different except when a spot is $a$, then the next spot must be 
	$b$ and not $c$. Thus, the equation is\\
	${2\choose1}{1\choose1}{4\choose1}{3\choose1}{2\choose1}{1\choose1}=192$
	\ee
\ee

% 5.2 # 6, 12, 15, 16, 20
\section*{5.2}
\be
% 6
\seti{5}
\item An urn has 3 red and 2 black balls. Two balls are removes at random without replacement. What is the probability that the 2 balls are
	\be
	% 6 a
	\item both red?\\
	$\paren{\frac{3}{5}}\times\paren{\frac{2}{4}} = \frac{3}{10} = 0.3$
	% 6 b
	\item both black?\\
	$\paren{\frac{2}{5}}\times\paren{\frac{1}{4}} = \frac{1}{10} = 0.1$
	% 6 c
	\item different colors?\\
	$\paren{\frac{2}{5}}\times\paren{\frac{3}{4}} + \paren{\frac{3}{5}}\times\paren{\frac{2}{4}}
	= \frac{3}{10} + \frac{3}{10} = 0.6$
	\ee
	
% 12
\seti{11}
\item Two 6 sided dice (one black, one red) are tossed. What is the probability that
	\be
	% 12 a
	\item the maximum of the numbers on the dice is 4?\\
	Probability of rolling a die less than or equal to 4 is $\frac{4}{6}$.
	The probability of both dice being less than or equal to 4 is 
	$\frac{4}{6}\times\frac{4}{6}=\frac{4}{9}=0.44$
	% 12 b
	\item the minimum of the numbers on the dice is 4?\\
	Probability of rolling a die greater than or equal to 4 is $\frac{3}{6}$.
	The probability of both dice being less than or equal to 4 is 
	$\frac{3}{6}\times\frac{3}{6}=\frac{1}{4}=0.25$
	% 12 c
	\item the product of the numbers on the dice is 4?\\
	The possible combinations for the product to be 4 is (1,4) and (2,2).
	The probability of both combinations is 
	$\frac{1}{6}\times\frac{1}{6}$ and the total is 
	$\frac{1}{6}\times\frac{1}{6}+\frac{1}{6}\times\frac{1}{6}=\frac{1}{18}=0.055$.
	\ee

% 15
\seti{14}
\item A fair coin is tossed 6 times. Find the probability of getting
	\be
	% 15 a
	\item no heads\\
		$\frac{1}{2^6}{6\choose0}=\frac{1}{64}$.
	% 15 b
	\item 1 head\\
		$\frac{1}{2^6}{6\choose1}=\frac{3}{32}$.
	% 15 c
	\item 2 heads\\
		$\frac{1}{2^6}{6\choose2}=\frac{15}{64}$.
	% 15 d
	\item 3 heads\\
		$\frac{1}{2^6}{6\choose3}=\frac{5}{16}$.
	% 15 e
	\item more than 3 heads\\
		$\frac{1}{2^6}\paren{{6\choose4}+{6\choose5}+{6\choose6}} 
		= \frac{1}{2^6}\paren{15+6+1}=\frac{22}{64}=\frac{11}{32}$.
	\ee
	
% 16
\item A fair coin is tossed until a head is obtained. What is the probability that the coin was tossed at least 4 times?\\
	The probability of the coin being tossed at least 4 times before a head is\\
	$\sum_{k=4}^\infty{\frac{1}{2^k}}\approx0.125$.

% 20
\seti{19}
\item 
	\be
	% 20 a
	\item A student answers a 3-question true-false test at random. What is the probability that she will get at least 
	two thirds of the questions correct?\\
		The chance she will get a least two thirds of the questions correct is equal to \vspace{0.25cm}\\
		$\frac{{3\choose2}+{3\choose3}}{2^3}=\frac{3+1}{8}=\frac{1}{2}$
	% 20 b
	\item Repeat part (a) for a 6-question test.\\
		The chance she will get a least two thirds of the questions correct is equal to \vspace{0.25cm}\\
		$\frac{{6\choose4}+{6\choose5}+{6\choose6}}{2^6}=\frac{15+6+1}{64}=\frac{11}{32}$
	% 20 c
	\item Repeat part (a) for a 9-question test.\\
		The chance she will get a least two thirds of the questions correct is equal to \vspace{0.25cm}\\
		$\frac{{9\choose6}+{9\choose7}+{9\choose8}+{9\choose9}}{2^9}=\frac{84+36+9+1}{512}=\frac{130}{512}$
	\ee

\ee


\end{document}















