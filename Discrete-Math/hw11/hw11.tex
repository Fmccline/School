\documentclass[12pt]{article}
\usepackage{setspace}  % To use linespacing
\usepackage{indentfirst} % Indents first line after sections
\usepackage{amssymb} % For \mathbb
\usepackage{enumerate} % For changing labels of enumerate
\usepackage[margin=1in]{geometry} % For editing margins
\usepackage{tikz} % Tikz drawing for graphs
\usetikzlibrary{arrows.meta} % Allows customizing arrows
\usetikzlibrary{backgrounds} % For framing a tikzpicture
\usetikzlibrary{calc, through}
\usetikzlibrary{decorations.markings}
\usetikzlibrary{arrows}
\usetikzlibrary{positioning}
\usepackage{amsmath}
\usepackage{ifthen}
\usepackage{intcalc} % \intcalcMod

% Make new commands
\newcommand{\N}{\mathbb{N}}
\newcommand{\R}{\mathbb{R}}
\newcommand{\Z}{\mathbb{Z}}
\newcommand{\abs}[1]{\left|#1\right|}
\newcommand{\paren}[1]{\left(#1\right)}
\newcommand{\fivespace}{\space\space\space\space\space}

\newcommand{\be}{\begin{enumerate}}
\newcommand{\ee}{\end{enumerate}}
\newcommand{\seti}[1]{\setcounter{enumi}{#1}}
\newcommand{\setii}[1]{\setcounter{enumii}{#1}}
\newcommand{\tand}{\text{ and }}

% Start main document
\begin{document}
\onehalfspacing
\hfill Frank Cline

\hfill Math 307

\hfill HW 11

% 5.1 # 4bcd (don't actually list them all), 6, 8, 9, 10, 13, 14, 16
% 5.2 # 6, 12, 15, 16, 20

\section*{5.1}

\be
% 4
\seti{3}
\item Let $A=\{1,2,3,4,5,6,7,8,9,10\}\tand B=\{2,3,5,7,11,13,17,19\}$.
	\be
	% 4 b
	\setii{1}
	\item How many subsets of A are there?\\
	There are 11 possible combinations: \\
	The empty set, set of length 1, set of length 2, ... , set of length 10\\
	$={10\choose0}\foreach \i in {1,2,3,4,5,6,7,8,9,10}{+{10\choose\i}}$\\
	$=1+10+45+120+210+252+210+120+45+10+1\\=1024$
	% 4 c
	\item How many 4-element subsets of A are there?\\
	The amount of 4-element subsets of A is equal to ${10\choose4} =210$.
	% 4 d
	\item How many 4-element subsets of A consist of 3 even and 1 odd number?\\
	Three of the elements must be even, and since there are 5 even numbers total in A, an even number is given by ${10\choose5}$.
	Since there are 3 even numbers, and numbers can't be repeated, that is ${10\choose5}{10\choose4}{10\choose3}$. The remaining 
	element must be odd, and there are 5 elements in A that are odd. Thus the remaining element must be ${10\choose5}$. 
	That makes the total to be ${10\choose5}{10\choose5}{10\choose4}{10\choose3}=834$.
	\ee

% 6
\seti{5}
\item A certain class consists of 12 men and 16 women. How many committees can be chosen from this class consisting of
	\be
	% 6 a
	\item 7 people?\\
	This is equal to $28\choose7$ because there are 28 people total, you need to choose 7 of them, and order doesn't matter.
	${28\choose7}=1,184,040$
	% 6 b
	\item 3 men and 4 women\\
	This is equal to $12\choose3$$16\choose4$ because you need to choose 3 out of 12 men and 4 out of 16 women, and order doesn't 
	matter. $12\choose3$$16\choose4$$=400400$
	% 6 c
	\item 7 women or 7 men\\
	This is equal to ${12\choose7}+{16\choose7}$ because there are two possibilities: choosing 7 men OR choosing 7 women where 
	order doesn't matter. ${12\choose7}+{16\choose7}=12,232$
	\ee

% 8
\seti{7}
\item How many committees consisting of 4 men and 4 women can be chosen from a group of 8 men and 6 women?\\
	This is equal to ${8\choose4}{6\choose4}$ because order doesn't matter and we're choosing 4 of 8 men, and 4 of 6 women.
	${8\choose4}{6\choose4}=85$.

% 9
\item Let $S=\{a,b,c,d\}\tand T=\{1,2,3,4,5,6,7\}$.
	\be
	% 9 a
	\item How many one-to-one functions are there from T into S?\\
	There are no one-to-one functions from T into S because T has more elements than S does, thus, it's impossible to map each 
	element in T to a unique element in S which is the definition of one-to-one.
	% 9 b
	\item How many one-to-one functions are there from S into T?\\
	The first element can map to one of 7 elements in T. The second element can then map to one of the 7 elements in T excluding 		the element one is mapped to, and so on. Thus the number of different one-to-one functions for S into T is 
	$(7)(7-1)(7-2)(7-3)=(7)(6)(5)(4)=840$.
	% 9 c
	\item How many functions are there from S into T?\\
	The number of functions from S into is equal to $(7)(7)(7)(7)$ because each of the 4 elements in S can map to any one of the 
	7 elements in T. $(7)(7)(7)(7)=2401$
	\ee

% 10
\item Let $P=\{1,2,3,4,5,6,7,8,9\}\tand Q=\{A,B,C,D,E\}$.
	\be
	% 10 a
	\item How many 4-element subsets of P are there?\\
	The number of 4-element subsets of P is equal to ${9\choose4}$ because you're choosing 4 elements from P and order doesn't 
	matter. ${9\choose4}=126$.
	% 10 b
	\item How many permutations of Q are there?\\
	There are 5 elements in Q, so the number of permutations of Q is equal to $5!=120$. 
	% 10 c
	\item How many license plates are there consisting of 3 letters from Q followed by 2 numbers from P? Repetition is allowed.\\
	The cases are:
		\begin{itemize}
		\item No letter rep: $P(5,3)$ \\
			Find the permutations of size 3 from the 5 letters.
		\item 2 letter rep: ${5\choose1}{4\choose1}{2\choose1}$ \\
			Choose 1 of the 5 letters, then choose another of the remaining 4 letters. The final letter must be the same as one
			of the first 2 letters.
		\item 3 letter rep: ${5\choose1}{1\choose1}{1\choose1}$\\	
			Choose 1 of the 5 letters, and the other 2 must be that letter.
		\item No num rep: $P(9,2)$\\
			Find the permutations of size 2 from the 9 numbers.
		\item 2 num rep: $9\choose1$\\
			Choose 1 of the numbers and they will be repeated.
		\end{itemize}
		The total number of possibilities\\
		$=[(\text{No letter rep})+(\text{2 letter rep})+(\text{3 letter rep})]\times[(\text{no num rep})+(\text{2 num rep})]$ \\
		$=[P(5,3)+(5)(4)(2)+(5)]\times[P(9,2)+9]$\\
		$=(60+40+5)\times(72+9)=(105)\times(81)$\\
		$=8505$.
	\ee

% 13
\seti{12}
\item 

\ee
\end{document}















